% Options for packages loaded elsewhere
\PassOptionsToPackage{unicode}{hyperref}
\PassOptionsToPackage{hyphens}{url}
%
\documentclass[
]{article}
\usepackage{amsmath,amssymb}
\usepackage{lmodern}
\usepackage{iftex}
\ifPDFTeX
  \usepackage[T1]{fontenc}
  \usepackage[utf8]{inputenc}
  \usepackage{textcomp} % provide euro and other symbols
\else % if luatex or xetex
  \usepackage{unicode-math}
  \defaultfontfeatures{Scale=MatchLowercase}
  \defaultfontfeatures[\rmfamily]{Ligatures=TeX,Scale=1}
\fi
% Use upquote if available, for straight quotes in verbatim environments
\IfFileExists{upquote.sty}{\usepackage{upquote}}{}
\IfFileExists{microtype.sty}{% use microtype if available
  \usepackage[]{microtype}
  \UseMicrotypeSet[protrusion]{basicmath} % disable protrusion for tt fonts
}{}
\makeatletter
\@ifundefined{KOMAClassName}{% if non-KOMA class
  \IfFileExists{parskip.sty}{%
    \usepackage{parskip}
  }{% else
    \setlength{\parindent}{0pt}
    \setlength{\parskip}{6pt plus 2pt minus 1pt}}
}{% if KOMA class
  \KOMAoptions{parskip=half}}
\makeatother
\usepackage{xcolor}
\IfFileExists{xurl.sty}{\usepackage{xurl}}{} % add URL line breaks if available
\IfFileExists{bookmark.sty}{\usepackage{bookmark}}{\usepackage{hyperref}}
\hypersetup{
  pdftitle={Tarea final},
  hidelinks,
  pdfcreator={LaTeX via pandoc}}
\urlstyle{same} % disable monospaced font for URLs
\usepackage[margin=1in]{geometry}
\usepackage{color}
\usepackage{fancyvrb}
\newcommand{\VerbBar}{|}
\newcommand{\VERB}{\Verb[commandchars=\\\{\}]}
\DefineVerbatimEnvironment{Highlighting}{Verbatim}{commandchars=\\\{\}}
% Add ',fontsize=\small' for more characters per line
\usepackage{framed}
\definecolor{shadecolor}{RGB}{248,248,248}
\newenvironment{Shaded}{\begin{snugshade}}{\end{snugshade}}
\newcommand{\AlertTok}[1]{\textcolor[rgb]{0.94,0.16,0.16}{#1}}
\newcommand{\AnnotationTok}[1]{\textcolor[rgb]{0.56,0.35,0.01}{\textbf{\textit{#1}}}}
\newcommand{\AttributeTok}[1]{\textcolor[rgb]{0.77,0.63,0.00}{#1}}
\newcommand{\BaseNTok}[1]{\textcolor[rgb]{0.00,0.00,0.81}{#1}}
\newcommand{\BuiltInTok}[1]{#1}
\newcommand{\CharTok}[1]{\textcolor[rgb]{0.31,0.60,0.02}{#1}}
\newcommand{\CommentTok}[1]{\textcolor[rgb]{0.56,0.35,0.01}{\textit{#1}}}
\newcommand{\CommentVarTok}[1]{\textcolor[rgb]{0.56,0.35,0.01}{\textbf{\textit{#1}}}}
\newcommand{\ConstantTok}[1]{\textcolor[rgb]{0.00,0.00,0.00}{#1}}
\newcommand{\ControlFlowTok}[1]{\textcolor[rgb]{0.13,0.29,0.53}{\textbf{#1}}}
\newcommand{\DataTypeTok}[1]{\textcolor[rgb]{0.13,0.29,0.53}{#1}}
\newcommand{\DecValTok}[1]{\textcolor[rgb]{0.00,0.00,0.81}{#1}}
\newcommand{\DocumentationTok}[1]{\textcolor[rgb]{0.56,0.35,0.01}{\textbf{\textit{#1}}}}
\newcommand{\ErrorTok}[1]{\textcolor[rgb]{0.64,0.00,0.00}{\textbf{#1}}}
\newcommand{\ExtensionTok}[1]{#1}
\newcommand{\FloatTok}[1]{\textcolor[rgb]{0.00,0.00,0.81}{#1}}
\newcommand{\FunctionTok}[1]{\textcolor[rgb]{0.00,0.00,0.00}{#1}}
\newcommand{\ImportTok}[1]{#1}
\newcommand{\InformationTok}[1]{\textcolor[rgb]{0.56,0.35,0.01}{\textbf{\textit{#1}}}}
\newcommand{\KeywordTok}[1]{\textcolor[rgb]{0.13,0.29,0.53}{\textbf{#1}}}
\newcommand{\NormalTok}[1]{#1}
\newcommand{\OperatorTok}[1]{\textcolor[rgb]{0.81,0.36,0.00}{\textbf{#1}}}
\newcommand{\OtherTok}[1]{\textcolor[rgb]{0.56,0.35,0.01}{#1}}
\newcommand{\PreprocessorTok}[1]{\textcolor[rgb]{0.56,0.35,0.01}{\textit{#1}}}
\newcommand{\RegionMarkerTok}[1]{#1}
\newcommand{\SpecialCharTok}[1]{\textcolor[rgb]{0.00,0.00,0.00}{#1}}
\newcommand{\SpecialStringTok}[1]{\textcolor[rgb]{0.31,0.60,0.02}{#1}}
\newcommand{\StringTok}[1]{\textcolor[rgb]{0.31,0.60,0.02}{#1}}
\newcommand{\VariableTok}[1]{\textcolor[rgb]{0.00,0.00,0.00}{#1}}
\newcommand{\VerbatimStringTok}[1]{\textcolor[rgb]{0.31,0.60,0.02}{#1}}
\newcommand{\WarningTok}[1]{\textcolor[rgb]{0.56,0.35,0.01}{\textbf{\textit{#1}}}}
\usepackage{graphicx}
\makeatletter
\def\maxwidth{\ifdim\Gin@nat@width>\linewidth\linewidth\else\Gin@nat@width\fi}
\def\maxheight{\ifdim\Gin@nat@height>\textheight\textheight\else\Gin@nat@height\fi}
\makeatother
% Scale images if necessary, so that they will not overflow the page
% margins by default, and it is still possible to overwrite the defaults
% using explicit options in \includegraphics[width, height, ...]{}
\setkeys{Gin}{width=\maxwidth,height=\maxheight,keepaspectratio}
% Set default figure placement to htbp
\makeatletter
\def\fps@figure{htbp}
\makeatother
\setlength{\emergencystretch}{3em} % prevent overfull lines
\providecommand{\tightlist}{%
  \setlength{\itemsep}{0pt}\setlength{\parskip}{0pt}}
\setcounter{secnumdepth}{-\maxdimen} % remove section numbering
\ifLuaTeX
  \usepackage{selnolig}  % disable illegal ligatures
\fi

\title{Tarea final}
\author{}
\date{\vspace{-2.5em}2023-07-21}

\begin{document}
\maketitle

\hypertarget{loguxedstica}{%
\section{Logística}\label{loguxedstica}}

Debes empujar tu trabajo final a la carpeta tarea\_final de la rama
tareas a más tardar el hasta las 23:59. Solo debes empujar código al
repositorio.

Tu envío debe ser una carpeta con tu nombre y apellidos (ej:
pedro\_muñoz\_gonzalez). La estructura de la carpeta debe ser la
siguiente:

\begin{itemize}
\tightlist
\item
  \emph{main.R}: script principal. Desde acá debes llamar a tus
  funciones
\item
  \emph{functions.R}: funciones que generes para hacer la tarea
\item
  .Rproj (obligatorio)
\end{itemize}

\hypertarget{encabezado}{%
\section{Encabezado}\label{encabezado}}

Tu jefatura te pide un reporte sobre algunos datos de la Encuesta
Suplementaria de Ingresos (ESI). Un requisito es que el trabajo sea lo
más reproducible posible y que sea autocontenido, es decir, que pueda
ejecutarse desde principio a fin por otros usuarios y usuarias de R.
Para esta tarea, \textbf{el uso de proyectos es obligatorio}.

Al realizar la tarea, ten en mente que idealmente tu trabajo debería
permitir que otra persona pueda hacer pull (o clonar) y ejecutar tu
código sin tener demasiados problemas. Incluir cosas de este tipo es una
buena idea y significa que estás siendo amable con tus colegas. Nota:
Una manera de mejorar considerablemente la reproducibilidad es utilizar
ambientes virtuales. Esto no es un requisito de la tarea y no se
considerará en la evaluación, pero te recomiendo revisar de qué se
trata. En R existen, al menos, 2 paquetes que implementan ambientes
virtuales: renv y packrat. Si decides trabajar con un ambiente virtual,
por favor, utiliza renv y no packrat.

\hypertarget{instrucciuxf3n-general}{%
\section{Instrucción general}\label{instrucciuxf3n-general}}

Considerando el volumen de datos, los ejercicios podrían resolverse
mediante soluciones manuales adhoc. Este tipo de soluciones no se
considerará correcto. Se espera que el código sea capaz de resolver la
tarea para n casos.

\hypertarget{ejercicio-1descargar-archivos}{%
\section{Ejercicio 1:descargar
archivos}\label{ejercicio-1descargar-archivos}}

Tu jefatura valora mucho la reproducibilidad, de modo que te entrega
algunas url, que deberás usar para descargar los datos desde la página
institucional del INE.

\begin{Shaded}
\begin{Highlighting}[]
\NormalTok{urls }\OtherTok{\textless{}{-}} \FunctionTok{c}\NormalTok{(}\StringTok{"https://www.ine.cl/docs/default{-}source/encuesta{-}suplementaria{-}de{-}ingresos/bbdd/csv\_esi/2021/esi{-}2021{-}{-}{-}personas.csv?sfvrsn=d03ae552\_4\&download=true"}\NormalTok{,}
\StringTok{"https://www.ine.cl/docs/default{-}source/encuesta{-}suplementaria{-}de{-}ingresos/bbdd/csv\_esi/2020/esi{-}2020{-}{-}{-}personas.csv?sfvrsn=fb1f7e0c\_4\&download=true"}\NormalTok{,}
\StringTok{"https://www.ine.cl/docs/default{-}source/encuesta{-}suplementaria{-}de{-}ingresos/bbdd/csv\_esi/2019/esi{-}2019{-}{-}{-}personas.csv?sfvrsn=9eb52870\_8\&download=true"}\NormalTok{,}
\StringTok{"https://www.ine.cl/docs/default{-}source/encuesta{-}suplementaria{-}de{-}ingresos/bbdd/csv\_esi/2018/esi{-}2018{-}{-}{-}personas.csv?sfvrsn=a5de2b27\_6\&download=true"}\NormalTok{,}
\StringTok{"https://www.ine.cl/docs/default{-}source/encuesta{-}suplementaria{-}de{-}ingresos/bbdd/csv\_esi/2017/1esi{-}2017{-}{-}{-}personas.csv?sfvrsn=d556c5a1\_6\&download=true"}\NormalTok{,}
\StringTok{"https://www.ine.cl/docs/default{-}source/encuesta{-}suplementaria{-}de{-}ingresos/bbdd/csv\_esi/2016/esi{-}2016{-}{-}{-}personas.csv?sfvrsn=81beb5a\_6\&download=true"}
\NormalTok{)}
\end{Highlighting}
\end{Shaded}

• Lo primero que haremos será encontrar los nombres de los archivos
dentro de las url. Utilizando expresiones regulares, crea un vector
llamado file\_names que almacene la porción de las url que contiene el
nombre de los archivos que luego descargaremos (ej:
esi-2021---personas.csv). Para esto, crea una función llamada
extract\_name. extract\_name debe recibir una url y devolver el nombre
del archivo. ¡No olvides utilizar purrr en tu solución para el vector de
nombres! • Crea una función llamada download\_esi\_data para descargar
un archivo. La función debe recibir 3 parámetros: url, file\_name y
directory. • Usando purrr, las url y el vector de nombres, descarga
todos los archivos en una carpeta llamada data en tu directorio de
trabajo. No está permitido el uso de rutas absolutas. Ejercicio 2: leer
archivos Ahora cargaremos los archivos en la sesión. Recuerda que no
necesariamente todos los archivos tienen el mismo separador. • Crea una
función llamada read\_esi\_data que lea un archivo. La función recibe
como argumento la ruta del archivo (ej: data/esi-2018---personas.csv).
read\_esi\_data debe ser capaz de reconocer el tipo de separador y leer
el archivo correctamente en todos los casos. Para lograr esto existen
varios caminos. Dependiendo de las versiones de tus dependencias, es
posible que tengas dificultades para usar readr::read\_csv. Si tienes
problemas, considera utilizar otras funciones para leer archivos
Ejercicio 3: obtener datos Tu jefatura está interesada en conocer
algunas características sobre las variables de diseño y sobre la
variable principal de ingresos (ing\_t\_p ). Para ello, te solicita lo
siguiente: • Tabla que contenga 3 columnas: version, n\_personas (idrph)
y n\_hogares (id\_identificacion). En la columna version debes usar la
siguiente estructura: esi\_\{año\}. Ejemplo: esi\_2017 • Tabla que
contenga mínimo, máximo, media, mediana, p10 y p90 del factor de
expansión (fact\_cal\_esi) para cada versión. Debes considerar una fila
por hogar (id\_identificacion) e incluir la columna version ¿Se observan
algunos pesos de muestreo atípicos? • Tabla que contenga el número de
estratos (estrato) con una sola unidad primaria de muestro
(conglomerado). Debes incluir la columna version. • Tabla que contenga
mínimo, máximo, media, mediana, p10 y p90 de los ingresos del trabajo
principal (ing\_t\_p) para cada versión. Esta tabla debe ser construida
a nivel persona, utilizando el factor de expansión (fact\_cal\_esi).
Ejercicio 4: mejorando el código Tu jefatura está muy satisfecha con los
resultados obtenidos, pero está preocupada por el tiempo de ejecución y
te pregunta si es posible crear un código más eficiente. Para ello, te
solicita comparar el tiempo de ejecución de algunas estrategias. Utiliza
el paquete microbenchmark, cuyo uso básico se describe aquí. Nota: No es
necesario que utilices más de 5 iteraciones para calcular el tiempo de
ejecución. Calcula el promedio de ingresos en las tablas de la ESI
(ing\_t\_p) mediante las siguientes estrategias: 1. Lista de tablas:
calcular promedio con herramientas de purrr (como en el ejercicio
anterior) 2 2. Tablas apiladas: calcular promedio con group\_by()
\%\textgreater\% summarise() (apila una tabla sobre otra en un
dataframe) 3. Lista de tablas: calcular promedio con herramientas de
purrr, utilizando una función creada por ti, que utilice data.table. 4.
Tablas apiladas: calcular promedio con data.table ¿Existen diferencias
importantes entre las distintas estrategias? ¿Hay alguna más eficiente
que otra? ¿Usar group\_by versus map hace alguna diferencia? Luego de
terminar, tu jefatura está tan complacida con tu trabajo que decide
regalarte un pasaje a Cancún a un hotel all-inclusive :)

\end{document}
